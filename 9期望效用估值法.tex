\section{期望效用估值法}
\begin{enumerate}[label=\arabic{section}.\arabic*]
    \item \sol
    \begin{align*}
        E[u(X_1)]&=1-\int_{0}^{+\infty} \e^{-x}\e^{-x}\,\d x=\frac{1}{2},\\
        E[u(X_2)]&=1-\int_0^2 \e^{-x}\frac{1}{2}\,\d x=\frac{1}{2}+\e^{-2},
    \end{align*}
    所以应该选择第二种投资方式.
    \item \sol\\ $E(X)=-0.4+0.1+0.25=-0.05<0$,所以$a=0$.
    \item \pro\\ 令$f(\alpha)=\ln x+p\ln(1+\alpha)+(1-p)\ln(1-\alpha)$,所以
    \[f'(\alpha)=\frac{p}{1+\alpha}-\frac{1+p}{1-\alpha}=-\frac{(2p+1)\alpha+1}{1-\alpha^2},\]
    若$\displaystyle p \leq \frac{1}{2}$,则$f'(\alpha) \leq 0$,则最优投资额为$\alpha x=0$.
    \item \pro\\ 令$f(\alpha)=\ln x+p\ln(1+r+\alpha-\alpha r)+(1-p)\ln(1+r)+(1-p)\ln(1-\alpha)$,所以
    \[f'(\alpha)=\frac{p(1-r)}{1+r+\alpha-\alpha r}-\frac{1-p}{1-\alpha}=\frac{(1-\alpha)(2p-1-r)}{(1+r+\alpha-\alpha r)(1-\alpha)},\]
    若$\displaystyle p \leq \frac{1}{2}$,则$f'(\alpha) \leq 0$,则最优投资额为$\alpha x=0$.
    \item \sol {\kaishu \textcolor{blue}{注意:这里翻译有误,是例9.3b.}}\\
    设$w_1=y,w_2=100-y$,则\[E[W]=100+0.15y+0.18(100-y)=118-0.03y,\]
    又由于$c(1,2)=\rho v_1v_2=0$,则\[\mathrm{Var}[W]=y^2(0.04)+(100-y)^2(0.0625)=0.1025y^2-12.5y+625,\]
    所以应该选择$y$,使下式的值达到最大:
    \[118-0.03y-0.005(0.1025y^2-12.5y+625)/2,\]
    易知$\displaystyle y=-\frac{-0.03+0.005\times12.5/2}{-2\times0.005\times0.1025/2}=2.439$,上式的值达到最大,即投资2.439于证券1,投资97.561于证券2.
    \item \sol\\
    设$w_1=y,w_2=100-y$,则\[E[W]=100+0.16y+0.18(100-y)=118-0.02y,\]
    又由于$c(1,2)=\rho v_1v_2=-0.02$,则\[\mathrm{Var}[W]=y^2(0.04)+(100-y)^2(0.0625)-2y(100-y)(0.02)=0.1425y^2-16.5y+625,\]
    所以应该选择$y$,使下式的值达到最大:
    \[118-0.02y-0.005(0.1425y^2-16.5y+625)/2,\]
    易知$\displaystyle y=\frac{0.02125}{0.0007125}=29.825$,上式的值达到最大. 最大期望效用为\[1-\exp\{-0.005[117.404-0.005(259.646)/2]\}=0.4422.\]
    \begin{enumerate}[label=\alph*)]
        \item $y=1$时,期望效用为
        \[1-\exp\{-0.005[116.98-0.005(608.6425)/2]\}=0.4386.\]
        \item $y=0$时,期望效用为
        \[1-\exp\{-0.005[118-0.005(625)/2]\}=0.4413.\]
    \end{enumerate}
    \item \pro\\
    因为\[W=w\sum_{i=1}^n \alpha_iX_i,\]
    若$U(x)=\ln x$,则
    \begin{align*}
        E[U(W)]&=E[\ln W]=E\left[\ln\left(w\sum_{i=1}^n \alpha_iX_i\right)\right]\\
        &=E\left[\ln w+\ln\left(\sum_{i=1}^n \alpha_iX_i\right)\right]\\
        &=\ln w+E\left[\ln\left(\sum_{i=1}^n \alpha_iX_i\right)\right]
    \end{align*}
    显然,$a_i$与$w$相互独立,即需要投资到各证券的财富比例不依赖于初始财富的数量.
    \item \pro
    \begin{enumerate}[label=\alph*)]
        \item \[U'(x)=ax^{a-1},U''(x)=a(a-1)x^{a-2},U'''(x)=a(a-1)(a-2)x^{a-3}>0,\]
        所以$U''(x)$关于$x$非减.
        \item \[U'(x)=b\e^{-bx},U''(x)=-b^2\e^{-bx},U'''(x)=b^3\e^{-bx}>0,\]
        所以$U''(x)$关于$x$非减.
        \item \[U'(x)=\frac{1}{x},U''(x)=-\frac{1}{x^2},U'''(x)=\frac{2}{x^3}>0,\]
        所以$U''(x)$关于$x$非减.
    \end{enumerate}
    \item \sol\\
    因为\[W=w\sum_{i=1}^n \alpha_iX_i,\]
    若$U(x)=\ln x$,则\[U[E(W)]-U''[E(W)]\mathrm{Var}(W)/2 = \ln[E(W)]-\frac{\mathrm{Var}(W)}{2[E(W)]^2}=\ln w+\ln \left[E\left(\sum\limits_{i=1}^n \alpha_iX_i\right)\right]-\frac{w^2\mathrm{Var}\left(\sum\limits_{i=1}^n \alpha_iX_i\right)}{2w^2E\left(\sum\limits_{i=1}^n \alpha_iX_i\right)},\]
    显然,$a_i$与$w$相互独立,即每个证券的财富比例不依赖于初始财富的数量.
    \item \sol {\kaishu \textcolor{blue}{注意:这里题干有误,是例9.3b.}}\\
    设$w_1=y,w_2=100-y$,则\[E[W]=100+0.15y+0.18(100-y)=118-0.03y,\]
    又由于$c(1,2)=\rho v_1v_2=-0.02$,则\[\mathrm{Var}[W]=y^2(0.04)+(100-y)^2(0.0625)-2y(100-y)(0.02)=0.1425y^2-16.5y+625,\]
    所以应该选择$y$,使下式的值达到最大:
    \[1-\exp[-0.005(118-0.03y)]-0.000025\exp[-0.005(118-0.03y)](0.1425y^2-16.5y+625)/2,\]
    可解得$y=16.408$时,上式的值达到最大.
    \item \sol
    \[P(W>g)=P\left(Z>\frac{g-E(W)}{\sqrt{\mathrm{Var}(W)}}\right),\]
    所以最大化$P(W>g)$即最小化$\displaystyle \frac{g-E(W)}{\sqrt{\mathrm{Var}(W)}}$,即最大化\[\frac{E(W)-g}{\sqrt{\mathrm{Var}(W)}}.\]
    \item \sol {\kaishu \textcolor{blue}{注意:这里翻译有误,是例9.3b.}}\\
    有
    \[E(W)=118-0.03y,\mathrm{Var}(W)=0.1425y^2-16.5y+625.\]
    \begin{enumerate}[label=\alph*)]
        \item 根据上一题,需最大化
        \[\frac{118-0.03y-110}{0.1425y^2-16.5y+625}=\frac{8-0.03y}{0.1425y^2-16.5y+625},\]
        可得$y=55.432$,所以投资55.432于证券1,投资44.568于证券2.
        \item 根据上一题,需最大化
        \[\frac{118-0.03y-115}{0.1425y^2-16.5y+625}=\frac{3-0.03y}{0.1425y^2-16.5y+625},\]
        可得$y=47.019$,所以投资47.019于证券1,投资52.981于证券2.
        \item 根据上一题,需最大化
        \[\frac{118-0.03y-120}{0.1425y^2-16.5y+625}=\frac{-2-0.03y}{0.1425y^2-16.5y+625},\]
        可得$y=5.363$,所以投资5.363于证券1,投资94.637于证券2.
        \item 根据上一题,需最大化
        \[\frac{118-0.03y-125}{0.1425y^2-16.5y+625}=\frac{-7-0.03y}{0.1425y^2-16.5y+625},\]
        可得$y=0$,所以投资0于证券1,投资100于证券2.
    \end{enumerate}
    \item \sol {\kaishu \textcolor{blue}{注意:这里题干有误,是例9.3d.}}\\
    利用软件等,可以解得\[\alpha=0.0763.\]
    \item \sol\\
    有$\beta_i=0.80,R_m=0.07$,则
    \begin{align*}
        r_f=0.05 & \Rightarrow R_i=r_f+\beta_i(R_m-r_f)=0.066,\\
        r_f=0.10 & \Rightarrow R_i=r_f+\beta_i(R_m-r_f)=0.076.
    \end{align*}
    \item \sol\\ $\displaystyle \beta=\sum\limits_{i=1}^k \alpha_i \beta_i$.
    \item \sol\\ 对比$R_i=r_f+\beta_i(R_m-r_f)$,显然CAPM是个单因素模型,且$a_i=(1-\beta_i)r_f,b_i=\beta_i,F=R_m$.
    \item \sol
    \begin{enumerate}[label=\alph*)]
        \item 能,因为$E(X_1+X_2)=2$,并由詹森不等式,风险厌恶者会更偏爱最终财富为2.
        \item 能,因为$E(2X_1)=E(X_1+X_2)=2,\mathrm{Var}(2X_1)=4>\mathrm{Var}(X_1+X_2)=2$.
        \item 不能,这取决于效用函数的具体表达式,$3X_1$同时具有更大的均值和方差.
        \item 因为
        \begin{align*}
            E[1-\e^{-X_1-X_2}]&=1-\e^{-2+2/2}=1-\e^{-1},\\
            E[1-\e^{3X_1}]&=1-\e^{-3+9/2}=1-\e^{1.5}<1-\e^{-1},
        \end{align*}
        所以选择$X_1+X_2$.
    \end{enumerate}
    \item \pro
    \begin{align*}
        \mathrm{Cov}(X_i,X_j)&=\mathrm{Cov}\left(\mu_i+\sum_{s=1}^na_{is}Z_s,\mu_j+\sum_{t=1}^na_{jt}Z_t\right)=\mathrm{Cov}\left(\sum_{s=1}^na_{is}Z_s,\sum_{t=1}^na_{jt}Z_t\right)\\
        &=\sum_{s=1}^n \sum_{t=1}^n \mathrm{Cov}\left(a_{is}Z_s,a_{jt}Z_t\right)=\sum_{r=1}^n \mathrm{Cov}\left(a_{ir}Z_r,a_{jr}Z_r\right)+\sum_{r=1}^n \sum_{r \neq k} \mathrm{Cov}\left(a_{ir}Z_r,a_{jk}Z_k\right)\\
        &=\sum_{r=1}^n a_{ir}a_{jr}\mathrm{Cov}\left(Z_r,Z_r\right)+\sum_{r=1}^n \sum_{r \neq k} a_{ir}a_{jk}\mathrm{Cov}\left(Z_r,Z_k\right)\\
        &=\sum_{r=1}^n a_{ir}a_{jr}
    \end{align*}
\end{enumerate}
\clearpage