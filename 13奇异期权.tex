\section{奇异期权}
\begin{enumerate}[label=\arabic{section}.\arabic*]
    \item \pro\\
    若$u \leq r$. 令时刻$y(y<t)$期权的价格为$s$. 相比在时刻$y$执行期权,在时刻$t$执行期权,在时刻$y$的期望收益为
    \[\e^{-r(t-y)}[s\e^{r(t-y)}-K\e^{ut}]=s-K\e^{ut-r(t-y)} \geq s-K\e^{uy},\]
    所以若$u \leq r$,那么永远不会提前执行这个看涨期权.
    \item \omitted
    \item \sol\\
    详细过程见书,$\displaystyle E(V)=S(0)\frac{1-\e^{r(n+1)/N}}{1-\e^{r/N}}$.
    \item \pro
    \begin{enumerate}[label=\alph*)]
        \item \begin{align*}
            \mathrm{Var}(W) & = \mathrm{Cov}\left(Y+\sum_{i=1}^n c_iX_i,Y+\sum_{j=1}^n c_jX_j\right)\\
            & = \mathrm{Cov}(Y,Y) + 2\mathrm{Cov}\left(Y,\sum_{j=1}^n c_jX_j\right)+\mathrm{Cov}\left(\sum_{i=1}^n c_iX_i,\sum_{j=1}^n c_jX_j\right)\\
            &=\mathrm{Var}(Y)+2\sum_{i=1}^n c_i\mathrm{Cov}(Y,X_i)+\sum_{i=1}^n\sum_{j=1}^n c_ic_j \mathrm{Cov}(X_i,X_j)\\
            &=\mathrm{Var}(Y)+2\sum_{i=1}^n c_i\mathrm{Cov}(Y,X_i)+\sum_{i=1}^nc_i^2 \mathrm{Cov}(X_i,X_i)+\sum_{i=1}^n\sum_{j \neq i} c_ic_j \mathrm{Cov}(X_i,X_j)\\
            &=\mathrm{Var}(Y)+\sum_{i=1}^nc_i^2 \mathrm{Var}(X_i)+2\sum_{i=1}^n c_i\mathrm{Cov}(Y,X_i)
        \end{align*}
        \item 两边同时对$c_i$求偏导,并让偏导等于0,可得
        \[2c_i\mathrm{Var}(X_i)+2\mathrm{Cov}(Y,X_i)=0 \Rightarrow c_i=-\frac{\mathrm{Cov}(Y,X_i)}{\mathrm{Var}(X_i)}.\]
    \end{enumerate}
    \item \omitted
    \item \omitted
    \item \sol\\
    与13.8节类似,只是等式变成了
    \[V_k(i)=\max\{su^id^{k-i} - K, pV_{k+1}(i + 1) + (1-p)W_{k+1}(i)\}.\]
    \item \sol\\
    障碍看涨期权的期望支付值以概率$p$乘以$u$,以概率$1-p$乘以$d$. 如果乘以$d$,则需要确认新的价格是否低于障碍值.
\end{enumerate}
\clearpage