\usepackage{ctex}                   % 支持中文
\usepackage{amsmath,amssymb}        % 支持数学公式
\usepackage{geometry}               % 排版页边距
\usepackage{pifont}                 % 带圈的数字
\usepackage{extarrows}              % 支持x箭头系列
\usepackage{mathrsfs}               % 支持大写花体字母
\usepackage{color}                  % 支持颜色
\usepackage{multirow}               % 支持合并单元格
\usepackage{graphicx}               % 支持插图
\usepackage{esint}                  % 支持环路积分符号
\usepackage{multicol}               % 支持分栏
\usepackage{listings}               % 支持代码块
\usepackage{xcolor}                 % 支持颜色
\usepackage{enumerate}              % 支持列举
\usepackage{enumitem}               % 支持列举
\usepackage{subfigure}              % 支持子图
\usepackage{caption}                % 支持给图加标题
\usepackage{url}                    % 支持输入网址
\usepackage{titlesec}               % 支持titleformat
\usepackage{float}                  % 支持分栏的时候插入图片的[H]
\usepackage{epstopdf}
\usepackage[hidelinks]{hyperref}    % 目录超链接功能,不加hidelinks会有红框
\usepackage[T1]{fontenc}            % 支持正文Times New Roman字体
\usepackage{mathptmx}               % 支持数学公式Times New Roman字体(更好看)

% 清除页眉,保留页脚的页码编号
\pagestyle{plain}
% 更改页边距
\geometry{left=19.1mm,right=19.1mm,top=25.4mm,bottom=25.4mm}
% 更改摘要的页边距
\makeatletter
\renewenvironment{abstract}{%
    \if@twocolumn
      \section*{\abstractname}%
    \else
      \small
      \begin{center}%
        {\bfseries \abstractname\vspace{-.5em}\vspace{\z@}}%
      \end{center}%
    \fi}
    {}
\makeatother
% 修改摘要的标题大小
\renewcommand{\abstractname}{{\large 摘要}\\}
% 更改section的自动编号方式、字体、居中,并设置为
\titleformat*{\section}{\centering\Large\bf}
\renewcommand\thesection{第\arabic{section}章}
% 更改网址url字体为空,即跟随主文字字体
\def\UrlFont{}
% 设置图片表格的caption为small,在本份文章中是五号
% \captionsetup{font={small}}
% 更改subsection的自动编号方式并设置字体大小
\titleformat*{\subsection}{\normalsize\bf}
\titleformat*{\subsubsection}{\normalsize\bf}
\renewcommand\thesubsection{\arabic{section}.\arabic{subsection}}
% 更改公式自动编号的方式
\renewcommand{\theequation}{\arabic{section}.\arabic{equation}}
% 代码块设置
\definecolor{mygreen}{rgb}{0,0.6,0}
\setmonofont{Consolas}
\lstset{language = matlab, numbers=left,
    breaklines=true, frame=false, numbersep=7pt,
    showspaces=false,
    columns=fullflexible,
    numberstyle=\tiny, keywordstyle=\color{blue!70},
    commentstyle=\color{mygreen},
    rulesepcolor=\color{red!0!green!0!blue!0}, basicstyle=\ttfamily,
    xleftmargin=1em, xrightmargin=1em, aboveskip=1em
}
% 更改公式自动编号的方式,并做到每重开一章,重新计数
\makeatletter
\@addtoreset{equation}{section}
\makeatother
\renewcommand{\theequation}{\arabic{section}.\arabic{equation}}
% 一个单元格内的内容自动换行
\newcommand{\tabincell}[2]{\begin{tabular}{@{}#1@{}}#2\end{tabular}}
% 更改参考文献的命名形式
\renewcommand{\refname}{\zhnum{section}、$\,\,$参考文献}
% 引用时变上标
\newcommand{\upcite}[1]{\textsuperscript{\textsuperscript{\cite{#1}}}}
% 定义
\newcommand{\sol}{{\heiti 解: }}
\newcommand{\pro}{{\heiti 证: }}
\newcommand{\omitted}{{\heiti 略.}}
\newcommand{\rank}{\mathrm{rank}}
\newcommand{\e}{\mathrm{e}}
\renewcommand{\d}{\mathrm{d}}
\newcommand{\eff}{\mathrm{eff}}