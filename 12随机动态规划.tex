\section{随机动态规划}
\begin{enumerate}[label=\arabic{section}.\arabic*]
    \item \sol
    \begin{enumerate}[label=\alph*)]
        \item 令$q(j)=1-p(j)$,所以
        \[V_k(n)=\begin{cases}
            0, & k>n\\\max\limits_{1 \leq j \leq n}\left\{p(j)V_{k-1}(n-j)+q(j)V_k(n-j)\right\}, & k \leq n.
        \end{cases}\]
        \item \begin{align*}
            & V_1(1) = p(1) = 0.2, \quad a_1(1) = 1,\\
            & V_1(2) = \max{p(1) + q(1)V_1(1), p(2)} = 0.4, \quad a_1(2) = 2,\\
            & V_1(3) = \max{p(1) + q(1)V_1(2), p(2) + q(2)V_1(1), p(3)} = 0.6, \quad a_1(3) = 3,\\
            & V_2(2) = p^2(1) = 0.04, \quad a_2(2) = 1,\\
            & V_2(3) = \max{p(1)V_1(2) + q(1)V_2(2), p(2)V_1(1)} = 0.112, \quad a_2(3) = 1,\\
            & V_2(4) = \max{p(1)V_1(3) + q(1)V_2(3), p(2)V_1(2) + q(2)V_2(2), p(3)V1(1)} = 0.2096, \quad a_2(4) = 1,
        \end{align*}
        所以最大概率为0.2096. 最优策略为:当仍然需要$j$台机器,并且还有剩余的$i$台机器需要建造时,先投资1台,然后再投资$a_j(i)$.
    \end{enumerate}
    \item \sol\\
    有$V (i, 0) = i, V (0, i) = 0$,则
    \begin{align*}
        V (1, 1) & = \max\left\{0, 0 + \frac{1}{2}\right\} = 1/2,\\
        V (2, 1) & = \max\left\{0, \frac{1}{3} + \frac{2}{3}V(1,1)+\frac{1}{3}V(2,0)\right\} = 4/3,\\
        V (1, 2) & = \max\left\{0, -\frac{1}{3} + \frac{1}{3}V(0,2)+\frac{2}{3}V(1,1)\right\} = 0,\\
        V (2, 2) & = \max\left\{0, 0 + \frac{1}{2}V(1,2)+\frac{1}{2}V(2,1)\right\} = 2/3,\\
        V (1, 3) & \leq V (1, 2) = 0, V (1, 4) \leq V (1, 3) = 0,\\
        V (2, 3) & = \max\left\{0, -\frac{1}{5} + \frac{2}{5}V(1,3)+\frac{3}{5}V(2,2)\right\} = 1/5,\\
        V (2, 4) & = \max\left\{0, -\frac{2}{6} + \frac{2}{6}V(1,4)+\frac{4}{6}V(2,3)\right\} = 0,\\
        V (3, 1) & = \max\left\{0, \frac{1}{2} + \frac{3}{4}V(2,1)+\frac{1}{4}V(3,0)\right\} = 9/4,\\
        V (3, 2) & = \max\left\{0, \frac{1}{5} + \frac{3}{5}V(2,2)+\frac{2}{5}V(3,1)\right\} = 3/2,\\
        V (3, 3) & = \max\left\{0, 0 + \frac{1}{2}V(2,3)+\frac{1}{2}V(3,2)\right\} = 17/20,\\
        V (3, 4) & = \max\left\{0, -\frac{1}{7} + \frac{3}{7}V(2,4)+\frac{4}{7}V(3,3)\right\} = 12/35.
    \end{align*}
    \item \pro\\
    可利用数学归纳法完成$V_n(x)$与$\alpha^*$的证明. 因为$\ln x$是凹函数,则
    \[E[\ln(\alpha Y+1-\alpha)] \leq \ln[\alpha E(Y)+1-\alpha] \leq \ln 1=0,\]
    所以$\alpha^*=0$.
    \item \sol {\kaishu \textcolor{blue}{注意:这里题干有误,应该是例12.2a.}}\\
    此时的最优策略就是求满足下列条件的$i$:
    \[i(1-\beta)>\beta E[(X-i)^+-c].\]
    \item \sol
    \begin{enumerate}[label=\alph*)]
        \item 在你赢了$k$局前,一定会赢$k-1$局. 所以最优策略就是以最低的期望成本赢$k-1$局,并在下一场游戏中投入$x$.
        \item 在以最低的期望成本赢$k-1$局并在下一场游戏中投入$x$后,赢$k-1$局的次数服从参数为$p(x)$的几何分布. 所以赢$k$局的最低的期望成本满足$\displaystyle V_k=\min\limits_{x \geq 1}\frac{V_{k-1}+x}{p(x)}$.
        \item 令$H(j)$为连续赢$j$局时,连续赢$n$局的最低期望成本. 于是有
        \[H(j)=\min_x\{x + p(x)H(j + 1) + (1 - p(x))V_n\}, j < n.\]
        从$j=n-1$开始,下一步是$j=n-2$,由此可以递归求解. 最小化等式右侧的$x$就是连续赢$j$局时的最佳投资金额.
    \end{enumerate}
    \item \sol
    \begin{enumerate}[label=\alph*)]
        \item 状态是当前已收到的赠券的种类;决策是是收集还是停止收集.
        \item 令$V(j)$是当前已收到的赠券的种类为$j$时,净回报的最大期望. 最优化函数为
        \[V(j)=\max\left\{jr,-1+\frac{j}{n}V(j)+\frac{n-j}{n}V(j+1)\right\}.\]
        \item 状态是$j$时的一阶前向策略为
        \[jr \geq -1+\frac{j}{n}jr+\frac{n-j}{n}(j+1)r,\]
        并且若$\displaystyle r\frac{n-j}{n}<1$则停止收集.
        \item 这是最优策略,因为状态无法减少,停止状态集是闭集.
        \item 状态是当前已收到的赠券的种类的子集.
        \item 记子集为$S$,若$\displaystyle r\sum_{i \notin S}p_i=1$,则一阶前向策略将会停止. 这是最优策略,因为当前已收到的赠券一定是一个人所有收到的赠券的子集.
    \end{enumerate}
    \item \sol\\
    当红球数小于或等于黑球数时,一阶前向策略将会停止. 这是一个很差的策略.
\end{enumerate}
\clearpage